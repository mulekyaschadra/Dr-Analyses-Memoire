% Options for packages loaded elsewhere
\PassOptionsToPackage{unicode}{hyperref}
\PassOptionsToPackage{hyphens}{url}
%
\documentclass[
]{book}
\usepackage{amsmath,amssymb}
\usepackage{lmodern}
\usepackage{ifxetex,ifluatex}
\ifnum 0\ifxetex 1\fi\ifluatex 1\fi=0 % if pdftex
  \usepackage[T1]{fontenc}
  \usepackage[utf8]{inputenc}
  \usepackage{textcomp} % provide euro and other symbols
\else % if luatex or xetex
  \usepackage{unicode-math}
  \defaultfontfeatures{Scale=MatchLowercase}
  \defaultfontfeatures[\rmfamily]{Ligatures=TeX,Scale=1}
\fi
% Use upquote if available, for straight quotes in verbatim environments
\IfFileExists{upquote.sty}{\usepackage{upquote}}{}
\IfFileExists{microtype.sty}{% use microtype if available
  \usepackage[]{microtype}
  \UseMicrotypeSet[protrusion]{basicmath} % disable protrusion for tt fonts
}{}
\makeatletter
\@ifundefined{KOMAClassName}{% if non-KOMA class
  \IfFileExists{parskip.sty}{%
    \usepackage{parskip}
  }{% else
    \setlength{\parindent}{0pt}
    \setlength{\parskip}{6pt plus 2pt minus 1pt}}
}{% if KOMA class
  \KOMAoptions{parskip=half}}
\makeatother
\usepackage{xcolor}
\IfFileExists{xurl.sty}{\usepackage{xurl}}{} % add URL line breaks if available
\IfFileExists{bookmark.sty}{\usepackage{bookmark}}{\usepackage{hyperref}}
\hypersetup{
  pdftitle={Malnutrition des enfants en âge scolaire dans la zone de santé de Kirotche},
  pdfauthor={Dr Lucien Bindu},
  hidelinks,
  pdfcreator={LaTeX via pandoc}}
\urlstyle{same} % disable monospaced font for URLs
\usepackage{color}
\usepackage{fancyvrb}
\newcommand{\VerbBar}{|}
\newcommand{\VERB}{\Verb[commandchars=\\\{\}]}
\DefineVerbatimEnvironment{Highlighting}{Verbatim}{commandchars=\\\{\}}
% Add ',fontsize=\small' for more characters per line
\usepackage{framed}
\definecolor{shadecolor}{RGB}{248,248,248}
\newenvironment{Shaded}{\begin{snugshade}}{\end{snugshade}}
\newcommand{\AlertTok}[1]{\textcolor[rgb]{0.94,0.16,0.16}{#1}}
\newcommand{\AnnotationTok}[1]{\textcolor[rgb]{0.56,0.35,0.01}{\textbf{\textit{#1}}}}
\newcommand{\AttributeTok}[1]{\textcolor[rgb]{0.77,0.63,0.00}{#1}}
\newcommand{\BaseNTok}[1]{\textcolor[rgb]{0.00,0.00,0.81}{#1}}
\newcommand{\BuiltInTok}[1]{#1}
\newcommand{\CharTok}[1]{\textcolor[rgb]{0.31,0.60,0.02}{#1}}
\newcommand{\CommentTok}[1]{\textcolor[rgb]{0.56,0.35,0.01}{\textit{#1}}}
\newcommand{\CommentVarTok}[1]{\textcolor[rgb]{0.56,0.35,0.01}{\textbf{\textit{#1}}}}
\newcommand{\ConstantTok}[1]{\textcolor[rgb]{0.00,0.00,0.00}{#1}}
\newcommand{\ControlFlowTok}[1]{\textcolor[rgb]{0.13,0.29,0.53}{\textbf{#1}}}
\newcommand{\DataTypeTok}[1]{\textcolor[rgb]{0.13,0.29,0.53}{#1}}
\newcommand{\DecValTok}[1]{\textcolor[rgb]{0.00,0.00,0.81}{#1}}
\newcommand{\DocumentationTok}[1]{\textcolor[rgb]{0.56,0.35,0.01}{\textbf{\textit{#1}}}}
\newcommand{\ErrorTok}[1]{\textcolor[rgb]{0.64,0.00,0.00}{\textbf{#1}}}
\newcommand{\ExtensionTok}[1]{#1}
\newcommand{\FloatTok}[1]{\textcolor[rgb]{0.00,0.00,0.81}{#1}}
\newcommand{\FunctionTok}[1]{\textcolor[rgb]{0.00,0.00,0.00}{#1}}
\newcommand{\ImportTok}[1]{#1}
\newcommand{\InformationTok}[1]{\textcolor[rgb]{0.56,0.35,0.01}{\textbf{\textit{#1}}}}
\newcommand{\KeywordTok}[1]{\textcolor[rgb]{0.13,0.29,0.53}{\textbf{#1}}}
\newcommand{\NormalTok}[1]{#1}
\newcommand{\OperatorTok}[1]{\textcolor[rgb]{0.81,0.36,0.00}{\textbf{#1}}}
\newcommand{\OtherTok}[1]{\textcolor[rgb]{0.56,0.35,0.01}{#1}}
\newcommand{\PreprocessorTok}[1]{\textcolor[rgb]{0.56,0.35,0.01}{\textit{#1}}}
\newcommand{\RegionMarkerTok}[1]{#1}
\newcommand{\SpecialCharTok}[1]{\textcolor[rgb]{0.00,0.00,0.00}{#1}}
\newcommand{\SpecialStringTok}[1]{\textcolor[rgb]{0.31,0.60,0.02}{#1}}
\newcommand{\StringTok}[1]{\textcolor[rgb]{0.31,0.60,0.02}{#1}}
\newcommand{\VariableTok}[1]{\textcolor[rgb]{0.00,0.00,0.00}{#1}}
\newcommand{\VerbatimStringTok}[1]{\textcolor[rgb]{0.31,0.60,0.02}{#1}}
\newcommand{\WarningTok}[1]{\textcolor[rgb]{0.56,0.35,0.01}{\textbf{\textit{#1}}}}
\usepackage{longtable,booktabs,array}
\usepackage{calc} % for calculating minipage widths
% Correct order of tables after \paragraph or \subparagraph
\usepackage{etoolbox}
\makeatletter
\patchcmd\longtable{\par}{\if@noskipsec\mbox{}\fi\par}{}{}
\makeatother
% Allow footnotes in longtable head/foot
\IfFileExists{footnotehyper.sty}{\usepackage{footnotehyper}}{\usepackage{footnote}}
\makesavenoteenv{longtable}
\usepackage{graphicx}
\makeatletter
\def\maxwidth{\ifdim\Gin@nat@width>\linewidth\linewidth\else\Gin@nat@width\fi}
\def\maxheight{\ifdim\Gin@nat@height>\textheight\textheight\else\Gin@nat@height\fi}
\makeatother
% Scale images if necessary, so that they will not overflow the page
% margins by default, and it is still possible to overwrite the defaults
% using explicit options in \includegraphics[width, height, ...]{}
\setkeys{Gin}{width=\maxwidth,height=\maxheight,keepaspectratio}
% Set default figure placement to htbp
\makeatletter
\def\fps@figure{htbp}
\makeatother
\setlength{\emergencystretch}{3em} % prevent overfull lines
\providecommand{\tightlist}{%
  \setlength{\itemsep}{0pt}\setlength{\parskip}{0pt}}
\setcounter{secnumdepth}{5}
\usepackage{booktabs}
\ifluatex
  \usepackage{selnolig}  % disable illegal ligatures
\fi
\usepackage[]{natbib}
\bibliographystyle{plainnat}

\title{Malnutrition des enfants en âge scolaire dans la zone de santé de Kirotche}
\author{Dr Lucien Bindu}
\date{2021-10-04}

\begin{document}
\maketitle

{
\setcounter{tocdepth}{1}
\tableofcontents
}
\hypertarget{about}{%
\chapter{About}\label{about}}

ce site est produit dans le cadre des analyses des données pour la rédaction du mémoire de master pour le troisème cycle en danté public

`

\hypertarget{presentation-des-donnuxe9es}{%
\chapter{Presentation des données}\label{presentation-des-donnuxe9es}}

Les donnés traités dans cette partie ont été récolté dur terrain par enquete dans la zone de santé de Kirotshe. au mois de septembre

\begin{verbatim}
## <labelled<double>[10]>: Sexe de l'enfant
##  [1] 1 1 1 1 1 2 1 2 1 1
## 
## Labels:
##  value    label
##      1 Masculin
##      2  Féminin
\end{verbatim}

\begin{Shaded}
\begin{Highlighting}[]
\FunctionTok{describe}\NormalTok{(df}\SpecialCharTok{$}\NormalTok{statut\_matrimonial)}
\end{Highlighting}
\end{Shaded}

\begin{verbatim}
## [272 obs.] statut matrimonial de la mère
## labelled double: 2 5 5 1 2 2 4 2 2 2 ...
## min: 1 - max: 5 - NAs: 3 (1.1%) - 6 unique values
## 5 value labels: [1] Célibataire [2] Marié [3] Divorcé [4] veuve [5] séparé
## 
##                   n     %  val%
## [1] Célibataire   4   1.5   1.5
## [2] Marié       222  81.6  82.5
## [3] Divorcé       5   1.8   1.9
## [4] veuve        23   8.5   8.6
## [5] séparé       15   5.5   5.6
## NA                3   1.1    NA
## Total           272 100.0 100.0
\end{verbatim}

\begin{Shaded}
\begin{Highlighting}[]
\NormalTok{df }\OtherTok{\textless{}{-}}\NormalTok{ df }\SpecialCharTok{\%\textgreater{}\%} \FunctionTok{select}\NormalTok{(}\SpecialCharTok{{-}}\NormalTok{type\_vaccin,}\SpecialCharTok{{-}}\NormalTok{methode\_planing,}\SpecialCharTok{{-}}\NormalTok{moment\_lavage)}
\end{Highlighting}
\end{Shaded}

Voici le tri à plat de la base des données, montrant les tendances générales pour toutes les variables d'étude.

\begin{verbatim}
## Table printed with `knitr::kable()`, not {gt}. Learn why at
## http://www.danieldsjoberg.com/gtsummary/articles/rmarkdown.html
## To suppress this message, include `message = FALSE` in code chunk header.
\end{verbatim}

\begin{tabular}{l|l}
\hline
**Variables d'étude** & **N = 272**\\
\hline
\_\_Numéro de la fiche\_\_ & \\
\hline
Moyenne  (ET) & 136  (79)\\
\hline
\_\_Adresse permanente(Village)\_\_ & \\
\hline
Birere & 2 (0.7\%)\\
\hline
BIRERE & 9 (3.3\%)\\
\hline
BISHANGE BURONG9 & 1 (0.4\%)\\
\hline
BISHANGE BURONGO & 4 (1.5\%)\\
\hline
BISHANGE BUSAKABA & 1 (0.4\%)\\
\hline
BISHANGE BUSAKARA & 4 (1.5\%)\\
\hline
BISHANGE CENTRE & 4 (1.5\%)\\
\hline
BULENGERO & 8 (2.9\%)\\
\hline
BURONGO & 1 (0.4\%)\\
\hline
BUSAKARA & 5 (1.8\%)\\
\hline
BUSHEKA & 3 (1.1\%)\\
\hline
BUSHUHE & 10 (3.7\%)\\
\hline
Kaduki & 2 (0.7\%)\\
\hline
KADUKI & 4 (1.5\%)\\
\hline
KADUKI STADE & 7 (2.6\%)\\
\hline
KAFEMBE & 1 (0.4\%)\\
\hline
KAMBA & 9 (3.3\%)\\
\hline
KARUBA & 1 (0.4\%)\\
\hline
KASHAKI MISHAVU & 1 (0.4\%)\\
\hline
KASURA & 9 (3.3\%)\\
\hline
KASURA/KAKOMERO & 1 (0.4\%)\\
\hline
KATEMBE & 7 (2.6\%)\\
\hline
KIBONDE & 10 (3.7\%)\\
\hline
KIDHWATI & 1 (0.4\%)\\
\hline
KING\_KAHIRA & 1 (0.4\%)\\
\hline
KINGI BUROHA & 1 (0.4\%)\\
\hline
KINGI KAHIRA & 13 (4.8\%)\\
\hline
KINGI\_KAHIRA & 5 (1.8\%)\\
\hline
Kishusha & 1 (0.4\%)\\
\hline
KISHWATI & 9 (3.3\%)\\
\hline
KISHWATSI & 4 (1.5\%)\\
\hline
KIZHWATI & 1 (0.4\%)\\
\hline
LULOBOGO & 1 (0.4\%)\\
\hline
MAJAGI & 1 (0.4\%)\\
\hline
MANYUTSA & 1 (0.4\%)\\
\hline
MANYUTSA I & 1 (0.4\%)\\
\hline
MATONGE & 4 (1.5\%)\\
\hline
MATUTSA I & 3 (1.1\%)\\
\hline
MAYUTSA & 5 (1.8\%)\\
\hline
MAYUTSA I & 7 (2.6\%)\\
\hline
MISHAVU & 6 (2.2\%)\\
\hline
Mitumbala & 3 (1.1\%)\\
\hline
MITUMBALA & 3 (1.1\%)\\
\hline
MONIMA & 1 (0.4\%)\\
\hline
MONUMENT & 2 (0.7\%)\\
\hline
MOSQUE & 10 (3.7\%)\\
\hline
MTUMBALA CENTRE & 8 (2.9\%)\\
\hline
MTUMBALA KAYIRENGE & 2 (0.7\%)\\
\hline
MTUMBALA MATOYE & 1 (0.4\%)\\
\hline
MUSHUNUNU & 4 (1.5\%)\\
\hline
MUSHUNUNU 1 & 1 (0.4\%)\\
\hline
MUSHWATI & 1 (0.4\%)\\
\hline
MUSHWATSI & 1 (0.4\%)\\
\hline
MUTUMBALA & 1 (0.4\%)\\
\hline
MUYUTSA I & 4 (1.5\%)\\
\hline
Ndobogo & 1 (0.4\%)\\
\hline
NDOBOGO & 1 (0.4\%)\\
\hline
RUSHOGA & 18 (6.6\%)\\
\hline
SHASHA 1 & 15 (5.5\%)\\
\hline
SHASHA 2 & 14 (5.1\%)\\
\hline
TWEGERANE & 11 (4.0\%)\\
\hline
TWIGERANE & 1 (0.4\%)\\
\hline
\_\_Aire de santé où se trouve le village\_\_ & \\
\hline
2 & 1 (0.4\%)\\
\hline
BIROHA KINGI & 1 (0.4\%)\\
\hline
BISHANGE & 11 (4.0\%)\\
\hline
BUHORA\_KINGI & 1 (0.4\%)\\
\hline
BULENGERA & 3 (1.1\%)\\
\hline
BULENGERO & 27 (9.9\%)\\
\hline
BUROHA-KINGI & 2 (0.7\%)\\
\hline
BUROHA KAHIR & 1 (0.4\%)\\
\hline
BUROHA KINGI & 12 (4.4\%)\\
\hline
BUSHUHE & 17 (6.2\%)\\
\hline
C S BISHANGE & 1 (0.4\%)\\
\hline
C.S BISHANGE & 8 (2.9\%)\\
\hline
C.S BUSHUHE & 3 (1.1\%)\\
\hline
Kaduki & 4 (1.5\%)\\
\hline
KADUKI & 25 (9.2\%)\\
\hline
KARUBA & 25 (9.2\%)\\
\hline
KASURA & 17 (6.2\%)\\
\hline
KAUBA & 1 (0.4\%)\\
\hline
KIDHWATI MUSHAKI & 1 (0.4\%)\\
\hline
KISHWATI MUSHAKI & 4 (1.5\%)\\
\hline
Mitumbala & 2 (0.7\%)\\
\hline
MITUMBALA & 6 (2.2\%)\\
\hline
MSHUNUNU MUSHAKI & 1 (0.4\%)\\
\hline
MTUMBALA & 14 (5.1\%)\\
\hline
MTUMBULA & 1 (0.4\%)\\
\hline
Mushaki & 2 (0.7\%)\\
\hline
MUSHAKI & 15 (5.5\%)\\
\hline
MUSHAKI MUSHUNUNU & 2 (0.7\%)\\
\hline
MUTUMBALA & 1 (0.4\%)\\
\hline
RUHOHA & 1 (0.4\%)\\
\hline
RUHOHA\_KAHIRA & 1 (0.4\%)\\
\hline
RUHORA\_KAHIRA & 1 (0.4\%)\\
\hline
SAKE & 30 (11\%)\\
\hline
SHASHA & 30 (11\%)\\
\hline
\_\_statut matrimonial de la mère\_\_ & \\
\hline
Célibataire & 4 (1.5\%)\\
\hline
Marié & 222 (83\%)\\
\hline
Divorcé & 5 (1.9\%)\\
\hline
veuve & 23 (8.6\%)\\
\hline
séparé & 15 (5.6\%)\\
\hline
Manquant & 3\\
\hline
\_\_Taille du ménage\_\_ & \\
\hline
Moyenne  (ET) & 8.13  (7.91)\\
\hline
Manquant & 5\\
\hline
\_\_Combien d'enfants d'âge scolaire(Primaires) avez-vous?\_\_ & \\
\hline
1 & 68 (25\%)\\
\hline
2 & 88 (33\%)\\
\hline
3 & 67 (25\%)\\
\hline
4 & 29 (11\%)\\
\hline
5 & 10 (3.7\%)\\
\hline
6 & 4 (1.5\%)\\
\hline
7 & 3 (1.1\%)\\
\hline
43 & 1 (0.4\%)\\
\hline
Manquant & 2\\
\hline
\_\_Avez-vous dejà été à l'école?\_\_ & \\
\hline
Oui & 161 (60\%)\\
\hline
Non & 107 (40\%)\\
\hline
Manquant & 4\\
\hline
\_\_Si oui quel niveau d'étude avez-vous?\_\_ & \\
\hline
Primaire & 93 (59\%)\\
\hline
Secondaire & 62 (39\%)\\
\hline
Universitaire & 3 (1.9\%)\\
\hline
Autre & 0 (0\%)\\
\hline
Manquant & 114\\
\hline
\_\_Autre\_\_ & NA  (NA)\\
\hline
Manquant & 272\\
\hline
\_\_Quelle est l'occupation de la mère?\_\_ & \\
\hline
Ménagère & 49 (18\%)\\
\hline
Commerçante & 66 (25\%)\\
\hline
Employé à une ONG & 2 (0.7\%)\\
\hline
Cultivatrice & 136 (51\%)\\
\hline
Emplpoyée de l'Etat & 9 (3.3\%)\\
\hline
Autre & 7 (2.6\%)\\
\hline
Manquant & 3\\
\hline
\_\_Autre\_\_ & \\
\hline
 & 271 (100\%)\\
\hline
Tailleur & 1 (0.4\%)\\
\hline
\_\_Quelle est l'occupation du père?\_\_ & \\
\hline
Ménagère & 11 (4.5\%)\\
\hline
Commerçante & 19 (7.7\%)\\
\hline
Employé à une ONG & 5 (2.0\%)\\
\hline
Cultivatrice & 117 (47\%)\\
\hline
Emplpoyée de l'Etat & 43 (17\%)\\
\hline
Autre & 52 (21\%)\\
\hline
Manquant & 25\\
\hline
\_\_Autre\_\_ & \\
\hline
 & 250 (92\%)\\
\hline
Éleveu & 1 (0.4\%)\\
\hline
Chauffeur & 1 (0.4\%)\\
\hline
Divorcé & 1 (0.4\%)\\
\hline
Militaire & 1 (0.4\%)\\
\hline
Motard & 11 (4.0\%)\\
\hline
MOTARD & 1 (0.4\%)\\
\hline
RAS & 1 (0.4\%)\\
\hline
Sciere & 2 (0.7\%)\\
\hline
Scieur & 2 (0.7\%)\\
\hline
Tailleur & 1 (0.4\%)\\
\hline
\_\_Quelles sont les habitudes alides alimentaires pendant la grossesse\_\_ & \\
\hline
Monotone & 44 (16\%)\\
\hline
Multivarié & 226 (83\%)\\
\hline
Autre & 1 (0.4\%)\\
\hline
Manquant & 1\\
\hline
\_\_Autre habitude\_\_ & NA  (NA)\\
\hline
Manquant & 272\\
\hline
\_\_Quels sont les interdits alimentaires de la mère pendant la grossessse\_\_ & \\
\hline
 & 129 (47\%)\\
\hline
Å’UFS & 1 (0.4\%)\\
\hline
À SIGNALER & 4 (1.5\%)\\
\hline
ALIMENTS AROMANTIQUES & 1 (0.4\%)\\
\hline
Aucun & 26 (9.6\%)\\
\hline
Aucune & 4 (1.5\%)\\
\hline
Choux et Avocat & 1 (0.4\%)\\
\hline
Feuille de manioc et patate douce & 1 (0.4\%)\\
\hline
Haricot & 1 (0.4\%)\\
\hline
Haricot patate douce & 1 (0.4\%)\\
\hline
Haricots frai & 1 (0.4\%)\\
\hline
L'huile & 1 (0.4\%)\\
\hline
La patte, pomme de terre & 1 (0.4\%)\\
\hline
Les haricots & 2 (0.7\%)\\
\hline
Oeufs & 1 (0.4\%)\\
\hline
Pad d'interdits & 1 (0.4\%)\\
\hline
Parfait & 1 (0.4\%)\\
\hline
Pas & 42 (15\%)\\
\hline
Pas d'interdits & 17 (6.2\%)\\
\hline
Pas des interdits & 2 (0.7\%)\\
\hline
Pas les interdits alimentaires & 1 (0.4\%)\\
\hline
Patate  douce & 1 (0.4\%)\\
\hline
Patate douce & 1 (0.4\%)\\
\hline
Plusieures & 1 (0.4\%)\\
\hline
Poisson, viande & 1 (0.4\%)\\
\hline
R A S & 1 (0.4\%)\\
\hline
R.A.S & 1 (0.4\%)\\
\hline
Rien & 17 (6.2\%)\\
\hline
RIEN & 1 (0.4\%)\\
\hline
Rien à signaler & 1 (0.4\%)\\
\hline
RIEN À SIGALER & 1 (0.4\%)\\
\hline
Riz & 2 (0.7\%)\\
\hline
RIZ, FARINE DE MAÏS & 1 (0.4\%)\\
\hline
Serpent & 1 (0.4\%)\\
\hline
Sombe, foufou & 1 (0.4\%)\\
\hline
Viande & 1 (0.4\%)\\
\hline
Viande de mouton & 1 (0.4\%)\\
\hline
\_\_Etes-vous un déplacé dans ce village de moins de 2 ans\_\_ & \\
\hline
Oui & 37 (14\%)\\
\hline
Non & 232 (86\%)\\
\hline
Manquant & 3\\
\hline
\_\_Quelles sont les causes de votre déplacement?\_\_ & \\
\hline
Guerre & 27 (82\%)\\
\hline
Aute & 6 (18\%)\\
\hline
Manquant & 239\\
\hline
\_\_Autre cause\_\_ & \\
\hline
 & 271 (100\%)\\
\hline
Violence & 1 (0.4\%)\\
\hline
\_\_Âge de l'enfant(mois)\_\_ & \\
\hline
Moyenne  (ET) & 89  (35)\\
\hline
Manquant & 4\\
\hline
\_\_Sexe de l'enfant\_\_ & \\
\hline
Masculin & 130 (48\%)\\
\hline
Féminin & 140 (52\%)\\
\hline
Manquant & 2\\
\hline
\_\_Lieu où s'est déroulé l'acouchement\_\_ & \\
\hline
à domicile & 31 (13\%)\\
\hline
FOSA & 205 (86\%)\\
\hline
Autres(à spécifier) & 1 (0.4\%)\\
\hline
Manquant & 35\\
\hline
\_\_taille\_enfant\_\_ & \\
\hline
Moyenne  (ET) & 124  (102)\\
\hline
Manquant & 10\\
\hline
\_\_poids\_enfant\_\_ & \\
\hline
Moyenne  (ET) & 24  (21)\\
\hline
Manquant & 4\\
\hline
\_\_perimetre\_enfant\_\_ & \\
\hline
Moyenne  (ET) & 305  (448)\\
\hline
Manquant & 15\\
\hline
\_\_Age gestationnel à l'acouchement\_\_ & \\
\hline
<9 mois & 11 (4.0\%)\\
\hline
9 mois & 227 (83\%)\\
\hline
>9mois & 29 (11\%)\\
\hline
Ne sait pas & 5 (1.8\%)\\
\hline
\_\_Votre enfant a-t-il été pésé à la naissance?\_\_ & \\
\hline
Oui & 192 (72\%)\\
\hline
Non & 75 (28\%)\\
\hline
Manquant & 5\\
\hline
\_\_poids\_enfant\_naissance\_\_ & \\
\hline
Moyenne  (ET) & 2,770  (1,625)\\
\hline
Manquant & 109\\
\hline
\_\_Votre enfant est-il vacciné?\_\_ & \\
\hline
Oui & 262 (97\%)\\
\hline
Non & 8 (3.0\%)\\
\hline
Manquant & 2\\
\hline
\_\_type\_vaccin1\_\_ & 218 (88\%)\\
\hline
Manquant & 23\\
\hline
\_\_type\_vaccin2\_\_ & 219 (88\%)\\
\hline
Manquant & 23\\
\hline
\_\_type\_vaccin3\_\_ & 210 (84\%)\\
\hline
Manquant & 23\\
\hline
\_\_type\_vaccin4\_\_ & 211 (85\%)\\
\hline
Manquant & 23\\
\hline
\_\_type\_vaccin5\_\_ & 119 (48\%)\\
\hline
Manquant & 23\\
\hline
\_\_type\_vaccin6\_\_ & 17 (6.8\%)\\
\hline
Manquant & 23\\
\hline
\_\_Autre type de vaccin\_\_ & NA  (NA)\\
\hline
Manquant & 272\\
\hline
\_\_L'enfant a-t-il été supplémenté en vitamine A?\_\_ & \\
\hline
Oui & 253 (93\%)\\
\hline
Non & 18 (6.6\%)\\
\hline
Manquant & 1\\
\hline
\_\_L'enfant a-t-il eu la fièvre durant les 6mois?\_\_ & \\
\hline
Oui & 133 (49\%)\\
\hline
Non & 139 (51\%)\\
\hline
\_\_l'enfant a-t-il eu une IRA durant les 6 mois passés?\_\_ & \\
\hline
Oui & 116 (43\%)\\
\hline
Non & 152 (57\%)\\
\hline
Manquant & 4\\
\hline
\_\_l'enfant a-t-il eu la rougeole l'année passée?\_\_ & \\
\hline
Oui & 10 (3.7\%)\\
\hline
Non & 262 (96\%)\\
\hline
\_\_L'enfant a-t-il eu des Oeudèmes?\_\_ & \\
\hline
Oui & 11 (4.1\%)\\
\hline
Non & 259 (96\%)\\
\hline
Manquant & 2\\
\hline
\_\_L'enfant a-t-il fait la diarrhée les 6mois passés?\_\_ & \\
\hline
Oui & 43 (16\%)\\
\hline
Non & 228 (84\%)\\
\hline
Manquant & 1\\
\hline
\_\_qui était consulté en premier lieu quand l'enfant tombe malade?\_\_ & \\
\hline
La FOSA & 215 (81\%)\\
\hline
Tradi Praticien & 27 (10\%)\\
\hline
Autre & 22 (8.3\%)\\
\hline
Manquant & 8\\
\hline
\_\_Autre\_\_ & \\
\hline
 & 256 (94\%)\\
\hline
À domicile & 1 (0.4\%)\\
\hline
Chambre de prière & 1 (0.4\%)\\
\hline
Domicile & 1 (0.4\%)\\
\hline
Pharmacie & 11 (4.0\%)\\
\hline
T³ à domicile & 1 (0.4\%)\\
\hline
Traitement à domicile & 1 (0.4\%)\\
\hline
\_\_Combien de repas que l'enfant mange par jour?\_\_ & \\
\hline
3 fois & 35 (13\%)\\
\hline
2 fois & 195 (73\%)\\
\hline
1 fois & 36 (14\%)\\
\hline
Manquant & 6\\
\hline
\_\_Combien de fois que l'enfant prend des fruits par semaine?\_\_ & \\
\hline
Chaque jour & 22 (8.2\%)\\
\hline
Une fois & 68 (25\%)\\
\hline
Aucune & 179 (67\%)\\
\hline
Manquant & 3\\
\hline
\_\_L'enfant reçoit-il une collation quand il va à l'école?\_\_ & \\
\hline
Oui & 35 (13\%)\\
\hline
Non & 232 (87\%)\\
\hline
Manquant & 5\\
\hline
\_\_espace\_enfant\_aine\_\_ & \\
\hline
Moyenne  (ET) & 7  (10)\\
\hline
Manquant & 61\\
\hline
\_\_Avez-vous allaité l'enfant après acouchement?\_\_ & \\
\hline
Oui & 269 (99\%)\\
\hline
Non & 3 (1.1\%)\\
\hline
\_\_Sinon quelles sont les raisons?\_\_ & NA  (NA)\\
\hline
Manquant & 272\\
\hline
\_\_Après combien de temps de l'acouchement avez-vous mis l'enfant au sein?\_\_ & \\
\hline
Dans 30 min & 223 (84\%)\\
\hline
Dans 1 heure & 32 (12\%)\\
\hline
Dans 24 heures & 11 (4.1\%)\\
\hline
Manquant & 6\\
\hline
\_\_Avez-vous donné un autre aliment/ liquide autre que le lait maternel avant 6mois?\_\_ & \\
\hline
Oui & 66 (24\%)\\
\hline
Non & 205 (76\%)\\
\hline
Manquant & 1\\
\hline
\_\_Si oui quel type de repas avez-vous donné??\_\_ & \\
\hline
L'eau & 19 (31\%)\\
\hline
Autre lait que le lait maternel & 4 (6.6\%)\\
\hline
Bouillie & 36 (59\%)\\
\hline
Autres(à spécifier) & 2 (3.3\%)\\
\hline
Manquant & 211\\
\hline
\_\_Avez-vous pressé et jeter le premier lait maternel?\_\_ & \\
\hline
Oui & 21 (7.7\%)\\
\hline
Non & 250 (92\%)\\
\hline
Manquant & 1\\
\hline
\_\_Si oui quelles sont les raisons?\_\_ & \\
\hline
Sale & 8 (73\%)\\
\hline
Crée des douleurs abdominales & 1 (9.1\%)\\
\hline
Autres(à spécifier) & 2 (18\%)\\
\hline
Manquant & 261\\
\hline
\_\_Autre\_\_ & NA  (NA)\\
\hline
Manquant & 272\\
\hline
\_\_Continuez-vous à allaiter après l'acouchement?\_\_ & \\
\hline
Oui & 235 (87\%)\\
\hline
Non & 36 (13\%)\\
\hline
Manquant & 1\\
\hline
\_\_Combien de temps allaitez-vous  dans 24h après l'acouchement?\_\_ & \\
\hline
Moyenne  (ET) & 14  (14)\\
\hline
Manquant & 102\\
\hline
\_\_Combien de mois avez-vous exclusivement allaité?\_\_ & \\
\hline
Moyenne  (ET) & 8  (6)\\
\hline
Manquant & 33\\
\hline
\_\_Combien de mois avez-vous  allaité l'enfant?\_\_ & \\
\hline
Moyenne  (ET) & 20  (10)\\
\hline
Manquant & 19\\
\hline
\_\_Qui a l'habitude de s'occuper de l'enfant?\_\_ & \\
\hline
Mère & 199 (74\%)\\
\hline
sœur & 40 (15\%)\\
\hline
Grand-mère & 14 (5.2\%)\\
\hline
Berceuse & 10 (3.7\%)\\
\hline
Autres(à spécifier) & 7 (2.6\%)\\
\hline
Manquant & 2\\
\hline
\_\_Age de la mère\_\_ & \\
\hline
Moyenne  (ET) & 33  (7)\\
\hline
Manquant & 5\\
\hline
\_\_Âge à la première grossesse\_\_ & \\
\hline
Moyenne  (ET) & 17.98  (3.25)\\
\hline
Manquant & 17\\
\hline
\_\_Quelles est la parité(Nombre de grossesses\_\_ & \\
\hline
Moyenne  (ET) & 6.45  (2.60)\\
\hline
Manquant & 26\\
\hline
\_\_Avez-vous fait la CPN?\_\_ & \\
\hline
Oui & 260 (96\%)\\
\hline
Non & 10 (3.7\%)\\
\hline
Manquant & 2\\
\hline
\_\_Si oui au quantième mois de grossesse avez-vous commencé ?\_\_ & \\
\hline
3 & 79 (31\%)\\
\hline
4 & 87 (34\%)\\
\hline
5 & 40 (16\%)\\
\hline
6 & 21 (8.2\%)\\
\hline
7 & 25 (9.8\%)\\
\hline
8 & 1 (0.4\%)\\
\hline
9 & 1 (0.4\%)\\
\hline
16 & 1 (0.4\%)\\
\hline
27 & 1 (0.4\%)\\
\hline
Manquant & 16\\
\hline
\_\_Combien de fois vous visitez les services CPN durant la grossesse?\_\_ & \\
\hline
1 & 17 (6.5\%)\\
\hline
2 & 31 (12\%)\\
\hline
3 & 63 (24\%)\\
\hline
4 & 131 (50\%)\\
\hline
5 & 16 (6.1\%)\\
\hline
6 & 3 (1.1\%)\\
\hline
Manquant & 11\\
\hline
\_\_Connaissez-vous la planification familiale?\_\_ & \\
\hline
Oui & 160 (61\%)\\
\hline
Non & 104 (39\%)\\
\hline
Manquant & 8\\
\hline
\_\_Avevez-vous déjà utilisé une méthode de planification?\_\_ & \\
\hline
Oui & 60 (38\%)\\
\hline
Non & 97 (62\%)\\
\hline
Manquant & 115\\
\hline
\_\_methode\_planing1\_\_ & 10 (17\%)\\
\hline
Manquant & 214\\
\hline
\_\_methode\_planing2\_\_ & 36 (62\%)\\
\hline
Manquant & 214\\
\hline
\_\_methode\_planing3\_\_ & 6 (10\%)\\
\hline
Manquant & 214\\
\hline
\_\_methode\_planing4\_\_ & 2 (3.4\%)\\
\hline
Manquant & 214\\
\hline
\_\_methode\_planing5\_\_ & 5 (8.6\%)\\
\hline
Manquant & 214\\
\hline
\_\_moment\_lavage1\_\_ & 193 (73\%)\\
\hline
Manquant & 9\\
\hline
\_\_moment\_lavage2\_\_ & 159 (60\%)\\
\hline
Manquant & 9\\
\hline
\_\_moment\_lavage3\_\_ & 135 (51\%)\\
\hline
Manquant & 9\\
\hline
\_\_moment\_lavage4\_\_ & 134 (51\%)\\
\hline
Manquant & 9\\
\hline
\_\_moment\_lavage5\_\_ & 136 (52\%)\\
\hline
Manquant & 9\\
\hline
\_\_moment\_lavage6\_\_ & 240 (91\%)\\
\hline
Manquant & 9\\
\hline
\_\_moment\_lavage7\_\_ & 16 (6.1\%)\\
\hline
Manquant & 9\\
\hline
\_\_Autre moment\_\_ & NA  (NA)\\
\hline
Manquant & 272\\
\hline
\_\_Quelle est votre principale source d'eau pour la boisson?\_\_ & \\
\hline
Puits & 0 (0\%)\\
\hline
Source non  protégé & 33 (12\%)\\
\hline
Source Protégée & 21 (7.8\%)\\
\hline
Robinet Privé & 13 (4.8\%)\\
\hline
Robinet Public & 196 (73\%)\\
\hline
Rivière & 5 (1.9\%)\\
\hline
Lac & 0 (0\%)\\
\hline
Autres(à spécifier) & 1 (0.4\%)\\
\hline
Manquant & 3\\
\hline
\_\_Combien de minutes faites-vous aller et retour pour puiser?\_\_ & \\
\hline
Moyenne  (ET) & 29  (40)\\
\hline
Manquant & 6\\
\hline
\_\_Traitez-vous de l'eau?\_\_ & \\
\hline
Oui & 36 (14\%)\\
\hline
Non & 229 (86\%)\\
\hline
Manquant & 7\\
\hline
\_\_Avez-vous de latrine?\_\_ & \\
\hline
Oui & 229 (87\%)\\
\hline
Non & 35 (13\%)\\
\hline
Manquant & 8\\
\hline
\_\_Quel type de latrine?\_\_ & \\
\hline
Privée & 192 (75\%)\\
\hline
Latrine partagée & 58 (23\%)\\
\hline
Autres(à spécifier) & 5 (2.0\%)\\
\hline
Manquant & 17\\
\hline
\_\_Autre type\_\_ & \\
\hline
 & 268 (99\%)\\
\hline
Chez le voisin & 2 (0.7\%)\\
\hline
Voisin & 2 (0.7\%)\\
\hline
\_\_Avez-vous une poubelle?\_\_ & \\
\hline
À ciel ouvert & 116 (43\%)\\
\hline
Avec couvercle & 8 (3.0\%)\\
\hline
Pas de couvercle & 53 (20\%)\\
\hline
Pas de pubelle & 88 (33\%)\\
\hline
Autres(à spécifier) & 5 (1.9\%)\\
\hline
Manquant & 2\\
\hline
\_\_Autre type\_\_ & \\
\hline
 & 269 (99\%)\\
\hline
Public & 1 (0.4\%)\\
\hline
Sac à poubelle & 1 (0.4\%)\\
\hline
Strou creusé & 1 (0.4\%)\\
\hline
\end{tabular}

  \bibliography{book.bib,packages.bib}

\end{document}
